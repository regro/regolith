%! Author = simon
%! Date = 4/14/2020

% the jinja2 template is just latex, as you can see, but it can
% fill in information coming from the variables sent by the
% navy_letter program.  These are a big dictionary called
% contents and a few other small things, like to and from.
% We can write simple code in here to do the rendering, so
% if we have a list of entries, it can be any length and we just
% wrap it in a for loop.  You can see the syntax below to get
% the idea.
\documentclass[letter,12pt]{letter}
\usepackage[utf8]{inputenc}
\usepackage{times}
\usepackage[margin=1.0in]{geometry}

\begin{document}

\hfill {{contents['date']}}
\begin{tabular}{p{0.5in}p{6in}}
To: & {{ to }} \\
From: & {{ fr }} \\[12pt]
Subj: & {{ contents['subject'] }} \\[12pt]
Ref: & {{ contents['refs'][0] }} \\


      & {{ ref }} \\


[12pt]
Encl: & {{ contents['encls'][0] }} \\


      & {{ encl }} \\


[12pt]

\end{tabular}


{{ loop.index }}.\ \  {{ para }}\par





\end{document}